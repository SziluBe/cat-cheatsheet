\wde{2.2.X Unit \& Counit} Given $\cat A \ladj{F}{G} \cat B$, for each $A \in \cat A$ we have a map $(A \map{\eta_A} GF(A)) = \overline{(F(A) \map{1} F(A))}$, dually for each $B \in \cat B$ we have a map $(FG(B) \map{\varepsilon_B} B) = \overline{(G(B) \map{1} G(B))}.$ These define natural transformations $\eta : 1_{\cat A} \to G \circ F$, $\varepsilon : F \circ G \to 1_{\cat B}$, called the \textbf{unit} and \textbf{counit} respectively.
\wl{2.2.2} Given an adjunction $F \dashv G$ with unit $\eta$ and counit $\varepsilon$, the triangles 
\begin{tikzcd}[cramped, sep=scriptsize]
	F & FGF \\
	& F
	\arrow["{F\eta}", from=1-1, to=1-2]
	\arrow["{{\varepsilon}{F}}", from=1-2, to=2-2]
	\arrow["{1_F}"', from=1-1, to=2-2]
\end{tikzcd}   
\begin{tikzcd}[cramped, sep=scriptsize]
	G & GFG \\
	& G
	\arrow["{{\eta}{G}}", from=1-1, to=1-2]
	\arrow["{G\varepsilon}", from=1-2, to=2-2]
	\arrow["{1_G}"', from=1-1, to=2-2]
\end{tikzcd} 
commute.
\wl{2.2.4} Let $\cat A \ladj{F}{G} \cat B$ be an adjunction, with unit $\eta$ and counit $\varepsilon$. Then $\overline{g} = G(g) \circ \eta_A$ for any $g: F(A) \to B$, and $\overline{f} = \varepsilon_B \circ F(f)$ for any $f : A \to G(B)$.
\wt{2.2.5} Take categories and functors $\cat A \isofuncs{F}{G} \cat B$. There is a one-to-one correspondence between: (a) adjunctions between $F$ and $G$ (with $F$ on the left and $G$ on the right); (b) pairs $(1_{\cat{A}} \map{\eta} GF, FG \map{\varepsilon} 1_{\cat{B}}$ of natural transformations satisfying the triangle identities (2.2.2).
\wc{2.2.7} An adjunction between ordered sets consists of order-pre-serving maps $A \isofuncs{f}{g} B$ such that $\forall a \in A, \forall b \in B, \ f(a) \le b \iff a \le g(b)$.
\we{2.2.8} An equivalence $(F, G, \eta, \varepsilon)$ of categories is not necessarily an adjunction (there is no reason for the triangle inequalities to hold), however it is true that $F$ is left adjoint to $G$ and $G$ is right adjoint to $F$.