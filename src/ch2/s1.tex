\wde{2.1.1 Adjoint} Let $\cat A \isofuncs{F}{G} \cat B$ be categories and functors. We say that $F$ is \textbf{left adjoint} to $G$ and $G$ is \textbf{right adjoint} to $F$, and write $F \adj G$, if $\cat B(F(A), B) \cong \cat A(A, G(B))$ (2.1) naturally in $A \in \cat A$ and $B \in \cat B$. An \textbf{adjunction} between $F$ and $G$ is a choice of natural isomorphism (2.1). `Naturally in $A \in \cat A$ and $B \in \cat B$' means that there is a specified bijection (2.1) for each $A \in \cat A$ and $B \in \cat B$, and that it satisfies a naturality axiom. 
\wde{2.1.1.X Transpose} Given objects $A \in \cat A$ and $B \in \cat B$, the correspondence (2.1) between maps $F(A) \to B$ and $A \to G(B)$ is denoted by a horizontal bar in both directions: $\left(F(A) \map{g} B\right) \mapsto \left(A \map{\bar{g}} G(B)\right)$, $\left(F(A) \map{\bar{f}} B\right) \mapsfrom \left(A \map{f} G(B)\right)$. So $\bar{\bar{f}} = f$ and $\bar{\bar{g}} = g$. We call $\bar{f}$ the \textbf{transpose} of $f$, and similarly for $g$. 
\wde{2.1.1.X Naturality Axiom} The naturality axiom has two parts: $\overline{(F(A) \map{g} B \map{q} B')} = (A \map{\bar{g}} G(B) \map{G(q)} G(B))$ (that is $\overline{q \circ g} = G(q) \circ \bar{g},\ \forall g,q$), and $\overline{(A' \map{p} A \map{f} G(B))} = (F(A') \map{F(p)} F(A) \map{\bar{f}} B)$ $\forall p,f$.
\wde{2.1.7 Initial/Terminal} Let $\cat A$ be a category. An object $I \in \cat A$ is \textbf{initial} if for every $A \in \cat A$, there is exactly one map $I \to A$. An object $T \in \cat A$ is \textbf{terminal} if for every $A \in \cat A$, there is exactly one map $A \to T$.
\wl{2.1.8} Let $I$ and $I'$ be initial objects of a category. Then there is a unique isomorphism $I \to I'$. In particular $I \cong I'$.
\we{2.1.9} Viewing functors $\mathbf{1} \to \cat A$ as objects of $\cat A$, a left adjoint $F$ to the unique functor $\cat A \to \mathbf{1}$ is an initial object of $\cat A$. Similarly a right adjoint $G$ to $\cat A \to \mathbf{1}$ is a terminal object of $\cat A$.